\documentclass{article}

\usepackage{amsthm}
\newtheorem{pro}{Proposal}
\newtheorem{defi}{Definition}
\newtheorem{thm}{Theorem}

\usepackage[utf8]{inputenc} 		% encodage des caracteres utilise (pour les caracteres accentues) -- non utilise ici.
%\usepackage[latin1]{inputenc} 		% autre encodage
\usepackage[french]{babel}		% pour une mise en forme "francaise"
\usepackage{amsmath,amssymb,amsthm}	% pour les maths
\usepackage{graphicx}			% pour inclure des graphiques
\usepackage{hyperref}			% si vous souhaitez que les references soient des hyperliens
\usepackage{color}			% pour ajouter des couleurs dans vos textes


\def \R {\mathbb R}					% definit un nouveau mot cle LateX. Ici, \R designera l'ensemble des reels
\newcommand \fonctionsContinues[2] {C^0(#1,#2)}		% Une nouvelle commande avec un argument


\title{Implementing the Core Chase for the Description Logic ALC}
\author{Maël Abily}	% pour les accents, on peut soit preciser l'encodage et utiliser des caracteres accentues, soit utilise \'e pour un e accent aigu, \`e pour un e accent grave, etc...

\begin{document}
\maketitle						% Genere le titre

%\tableofcontents					% si on veut une table des matieres



\section{Introduction}

\begin{defi} We define a \underline{first-order language} as a set of constants (often noted a,b,c,$c_{1}$,...), predicates (P,Q,R,$P_{1}$,...) and variables (x,y,$x_{1}$,...). \\
A \underline{term} is a variable or a constant (often noted t,$t_{1}$,...).
We note \underline{Ar(P)} the arity of P. \\
$P(t_{1},...,t_{n})$ is an \underline{atom}. \\
A \underline{fact} is a variable-free atom. \\
A \underline{factbase} F := $\exists x_{1},...,x_{n}.P_{1}(t_{1}^{1},...,t_{k_{1}}^{1})\land ...\land P_{m}(t_{1}^{m},...,t_{k_{m}}^{m})$ is a existentially quantified conjunctions of atoms which is closed (it means that every variable of F is quantified). \underline{var(F)} (respectively \underline{cst(F)} and \underline{term(F)} is the set of variables (resp. constants and terms) that occur in F.
\end{defi}
\noindent \textbf{Remark} We will often see factbases as sets of atoms. For example, the factbase $\exists x,x_{1},x_{2},x_{3}. P(x) \land Q(x,a) \land R(x_{1},x_{2},x_{3},b)$ can be represented by \\
$\{P(x),Q(x,a),R(x_{1},x_{2},x_{3},b)\}$.

\begin{defi}
A \underline{substitution} $\sigma:X \to Terms$ is a function where X is a set of variables. For example $\{x \mapsto z, y \mapsto a \}$ is a substitution from \{x,y\} to Terms. \\
A \underline{homomorphism} from F to F' is a substitution $\sigma:var(F) \to term(F')$ where $\sigma(F) \subseteq F'$
\end{defi}
\begin{pro}
A factbase F entails a factbase F' (often noted F $\to F'$) $\Leftrightarrow$ there exists a homomorphism from F' to F. \\
For example, F = \{P(b,a),Q(x)\} entails F' = \{P(x,a)\} thanks to the homomorphism $\{x \mapsto b\}$ 
\end{pro}

\begin{defi}
An \underline{isomorphism} is a bijective homomorphism. \\
A subset F' $\subseteq F$ is a \underline{retract} of F if there exists a subsitution $\sigma$ such that $\sigma(F) = F'$ and $\sigma_{|F'}=id$ ($\sigma$ is called a \underline{retractation} from F to F'). \\
A factbase is a \underline{core} if its strict subset are not retracts. \\
A \underline{core} of a factbase F is a minimal subset of F that is a core.
\end{defi}

\begin{pro}
A factbase F is a core $\Leftrightarrow$ every homomorphism $\sigma: F \to F$ is a bijection.
\end{pro}

\noindent \textbf{Example} F = $\{R(a,x)\}$ is the core of F' = $\{R(a,x),R(y,z)\}$

\begin{defi}
An \underline{(existential) rule} R is a first-order formula	 of the form : $\forall \vec x.\forall \vec y. A(\vec x,\vec y) \rightarrow \exists \vec z. B(\vec x,\vec z)$ where  A and B are conjonctions of atoms. We define \underline{body(R)} = A, \underline{head(R)} = B and the frontier of R \underline{fr(R)} = $\vec x$ (the set of variables shared by the body and the head of R). \\
An \underline{ontology} O is a pair (T,F) where T is a set of existential rules and F is a factbase.
\end{defi}

\begin{defi}
Let T be a ruleset, $\alpha$ be a rule, $\sigma$ be a subsitution and F be a factbase. \\
The tuple $(\alpha,\sigma)$ is a \underline{trigger} for F if : 
\begin{itemize}
\item the domain of $\sigma$ is the set of all variables occurring in Body($\alpha$).
\item $\sigma(Body(\alpha)) \subseteq F$.
\item For all $\hat \sigma$ which extends $\sigma$ and has its domain equals to the set of all variables occurring in Body($\alpha$) and Head($\alpha$) ,$\hat \sigma(Head(\alpha)) \nsubseteq F$
\end{itemize}
\underline{$Tr_{\alpha}(F)$} is the set of all trigers $(\alpha,\sigma)$ for F. \\
\underline{$Tr_{T}(F)$} is the set of all trigers for F. \\
The rule $\alpha$ is \underline{satisfied} by F if $Tr_{\alpha}(F)$ = $\O$. \\
T is \underline{satisfied} by F if $Tr_{T}(F)$ = $\O$.
\end{defi}

\noindent \textbf{Example} If $\alpha = A(x,y) \rightarrow B(x,z)$, F = $\{A(b,c)\}$ and $\sigma = \{x \mapsto b, y \mapsto c \}$ then $(\alpha,\sigma)$ is a trigger for F.

\begin{defi}
a ruleset T is \underline{satisfied} by a factbase M if every rule of T is satisfied by M. \\
A factset M is a \underline{model} for an ontology O = (T,F) if $F \subset M$ and T is satisfied by M. \\
A \underline{Boolean conjunctive query (BCQ)} is a closed formula of the form \\ $\exists x_1,...,x_n. F(x_1,...,x_n)$ where F is a conjunction of atoms. \\
A fact set F \underline{entails} a BCQ $B = \exists x_1,...,x_n.F(x_1,...,x_n)$ (noted $F \vDash B$) if there exists a substitution $\sigma$ such that $\sigma(B) \subset F$. An ontology O \underline{entails} a BCQ $B = \exists x_1,...,x_n.F(x_1,...,x_n)$ (noted $O \vDash B$) if for every model M of O, $M \vDash B$.\\
A model U for an ontology O is \underline{universal} if for
every model M of O, there exists an homomorphism $h : U \to M$.
\end{defi}


\end{document}