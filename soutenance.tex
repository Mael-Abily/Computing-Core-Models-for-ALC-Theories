\documentclass{beamer}

\usepackage{amsthm}


\setlength{\parskip}{0.5em}

\usepackage[linesnumbered,ruled,vlined]{algorithm2e}
\usepackage{comment}
\usepackage{pgf}
\usepackage{tikz}
\usetikzlibrary{arrows,automata}
\usepackage[utf8]{inputenc} 		% encodage des caracteres utilise (pour les caracteres accentues) -- non utilise ici.
%\usepackage[latin1]{inputenc} 		% autre encodage
\usepackage[english]{babel}		% pour une mise en forme "anglaise"
\usepackage{amsmath,amssymb,amsthm}	% pour les maths
\usepackage{graphicx}			% pour inclure des graphiques
\usepackage{hyperref}			% si vous souhaitez que les references soient des hyperliens
\usepackage{color}			% pour ajouter des couleurs dans vos textes
\usepackage{todonotes}

\def \N {\mathbb N}	

\newcommand{\TodoDavid}[1]{\todo[color=green!40]{#1}}
\newcommand{\TodoJF}[1]{\todo{#1}}

\newcommand{\Vars}{\textbf{Vars}}
\newcommand{\Terms}{\textbf{Terms}}
\newcommand{\Preds}{\textbf{Preds}}
\newcommand{\Csts}{\textbf{Csts}}
\newcommand{\Merge}{\textit{Merge}}
\newcommand{\Depth}{\textit{depth}}
\newcommand{\Appl}{\textbf{appl}}
\newcommand{\father}{\textbf{father}}
\newcommand{\Tree}{\textit{Tree}}

\title{Implementing the Core Chase for the Description Logic ALC}

\author{Maël Abily}
\institute{INRIA}
\date{2021}

\begin{document}

\frame{\titlepage}

\begin{frame}
\frametitle{Factbase}
We consider variables $(x,y,\ldots)$, constants $(a,b,\textit{Pierre},\ldots)$ and predicates $(P,R,\textit{IsTheSonOf},\ldots)$

\emph{term} = variable or constant.

An \emph{atom} is of the form $P(t_1,\ldots,t_n)$ and is ground if $t_1,\ldots,t_n$ are constants.

Example: $\textit{IsTheSonOf(Pierre,Francis)}$ is a ground atom.

\emph{Factbase} = existentially closed formula of the form $\exists x_{1},\ldots,x_{n}.P_{1}(t_{1}^{1},\ldots,t_{k_{1}}^{1})\land \ldots\land P_{m}(t_{1}^{m},\ldots,t_{k_{m}}^{m})$ identify with:

$\{P_{1}(t_{1}^{1},\ldots,t_{k_{1}}^{1}), \ldots, P_{m}(t_{1}^{m},\ldots,t_{k_{m}}^{m})\}$ 
\end{frame}

\begin{frame}
\frametitle{Homomorphism}

A \emph{substitution} $\sigma:X \to \Terms$ is a function where X is a set of variables. For example $\{x \mapsto z, y \mapsto a \}$ is a substitution from $\{x,y\}$ to \Terms.
\begin{itemize}
\item if $c \in \Csts$, then $\sigma(c) = c$;
\item if $x \in \Vars \setminus X$, $\sigma(x) = x$;
\item if $f = P(t_1,\ldots,t_n)$ is an atom, then $\sigma(f) = P(\sigma(t_1),\ldots,\sigma(t_n))$; and
\item if $F = \{f_1,\ldots,f_n\}$ is a factbase, then $\sigma(F) = \{\sigma(f_1),\ldots,\sigma(f_n)\}$.
\end{itemize}

For two factbases $F$ and $F'$, a \emph{homomorphism} from $F$ to $F'$ is a substitution $\sigma:\Vars(F) \to \Terms$ where $\sigma(F) \subseteq F'$. 

An \emph{isomorphism} $h$ from $F$ to $F'$ is a bijective homomorphism where its inverse is a homomorphism from $F'$ to $F$. 


\end{frame}


\begin{frame}
\frametitle{Entailment}

A factbase $F$ \emph{entails} another factbase $Q$ if there exists a homomorphism from $Q$ to $F$.

Example: The factbase $F = \{P(b,a),A(x)\}$ entails the factbase $Q = \{P(x,a),P(y,z)\}$ due to the homomorphism $\{x \mapsto b, y \mapsto b, z \mapsto a \}$.

A factbase $G$ is a \emph{retract} of another factbase $F$ if $G \subseteq F$ and $G \models F$.


If a factbase $F$ does not contain a strict retract, then we say that $F$ is a \emph{core}. A \emph{core} of a factbase $F$ (noted \emph{$\textit{core}(F)$}) is a subset of $F$ that is a core.

The cores of a finite factbase F are unique up to isomorphism.


\end{frame}

\begin{frame}
\frametitle{The conjunctive query entailment}

\end{frame}



\end{document}