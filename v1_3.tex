\documentclass{article}

\usepackage{amsthm}
\theoremstyle{proposal}
\newtheorem{proposal}{Proposal}[section]
\theoremstyle{definition}
\newtheorem{definition}{Definition}[section]
\theoremstyle{theorem}
\newtheorem{theorem}{Theorem}[section]
\theoremstyle{example}
\newtheorem{example}{Example}[section]
\theoremstyle{remark}
\newtheorem{remark}{Remark}[section]


\usepackage[utf8]{inputenc} 		% encodage des caracteres utilise (pour les caracteres accentues) -- non utilise ici.
%\usepackage[latin1]{inputenc} 		% autre encodage
\usepackage[english]{babel}		% pour une mise en forme "francaise"
\usepackage{amsmath,amssymb,amsthm}	% pour les maths
\usepackage{graphicx}			% pour inclure des graphiques
\usepackage{hyperref}			% si vous souhaitez que les references soient des hyperliens
\usepackage{color}			% pour ajouter des couleurs dans vos textes
\usepackage{todonotes}

\def \N {\mathbb N}	
\def \R {\mathbb R}					% definit un nouveau mot cle LateX. Ici, \R designera l'ensemble des reels
\newcommand \fonctionsContinues[2] {C^0(#1,#2)}		% Une nouvelle commande avec un argument


\title{Implementing the Core Chase for the Description Logic ALC}
\author{Maël Abily}	% pour les accents, on peut soit preciser l'encodage et utiliser des caracteres accentues, soit utilise \'e pour un e accent aigu, \`e pour un e accent grave, etc...

\begin{document}
\maketitle						% Genere le titre

%\tableofcontents					% si on veut une table des matieres

I modified : definition of atom, the first remark, definition 2 and 3 (including the definition of core),exemple of core, proof of proposal 1.2, the def of $Tr_{T}(F)$, the last exemple and the def of universal model. \\
The goal is to answer a query with a given database  by computing a universal model. We are dealing with a restriction of FOL (Horn-ALC axioms).

\section{Introduction}

\subsection{Syntax}

\begin{definition}[first-order language, constants, predicates, variables, terms, arity, atoms, facts, factbases] We define a \emph{first-order language} as a set of constants (often noted a,b,c,$c_{1}$,...), predicates (P,Q,R,$P_{1}$,...) and variables (x,y,$x_{1}$,...). A \emph{term} is a variable or a constant (often noted t,$t_{1}$,...). We note \emph{Ar(P)} the arity of the predicate P. If $t_1,...,t_n$ are \textit{Terms} and P is a predicate where Ar(P) = n, $P(t_{1},...,t_{n})$ is an \emph{atom}. A \emph{fact} is a variable-free atom. Let $t_i^j$ be \textit{Terms} and $P_i$ be predicates. A \emph{factbase} F := $\exists x_{1},...,x_{n}.P_{1}(t_{1}^{1},...,t_{k_{1}}^{1})\land ...\land P_{m}(t_{1}^{m},...,t_{k_{m}}^{m})$ is an existentially quantified conjunctions of atoms which is closed (it means that every variable of F is quantified). \emph{var(F)} (respectively \emph{cst(F)} and \emph{term(F)}) is the set of variables (resp. constants and Terms) that occur in F. \emph{var} (respectively \emph{cst} and \emph{terms}) is the set of variables (resp. constants and Terms). \emph{Factbases} is the set of factbases.
\end{definition}
\noindent \textbf{Remark} We identify factbases as sets of atoms. For example, the factbase $\exists x,x_{1},x_{2},x_{3}. P(x) \land Q(x,a) \land R(x_{1},x_{2},x_{3},b)$ is represented by \\
$\{P(x),Q(x,a),R(x_{1},x_{2},x_{3},b)\}$.

\begin{definition}[substitution, homomorphism]
A \emph{substitution} $\sigma:X \to \textit{Terms}$\todo{Use the latex command textit around \textit{Terms}.}\ is a function where X is a set of variables. For example $\{x \mapsto z, y \mapsto a \}$ is a substitution from \{x,y\} to \textit{Terms}. We define $\sigma_1:X \cup \emph{const} \to \textit{Terms}$ :
\begin{itemize}
\item if x $\in X$, $\sigma_1(x) = \sigma(x)$;
\item if c is a constant, $\sigma_1(c) = c$;
\end{itemize}
We define $\sigma_2:\{$Factbases F where var(F) $\subseteq X\} \to \textit{Factbases}$ :
\begin{itemize}
\item if $P(t_{1},...,t_{n})$ is an atom in F and $t_1,...,t_n$ are variables in X or are constants $\sigma_2(\{P(t_{1},...,t_{n})\}) = \{P(\sigma_1(t_{1}),...,\sigma_1(t_{n}))\}$ ;
\item if $A_{1},...,A_{n}$ are atoms in F, $\sigma_2(\{A_{1},...,A_{n}\}) = \{\sigma_2(\{A_{1}\}),...,\sigma_2(\{A_{n}\})\}$.
\end{itemize}
Let F and F' be two factbases. A \emph{homomorphism} from F to F' is a substitution $\sigma:var(F) \to term(F')$ where $\sigma_2(F) \subseteq F'$. 
\end{definition}

\noindent \textbf{Remark} We identify $\sigma_2$ with $\sigma$.

\subsection{core}

\begin{definition}[identity, isomorphism, retract, core]
$id_{|F}$ is the substitution identity defined by : for all variable x $\in var(F)$, $id(x) = x$. An \emph{isomorphism} h is a bijective homomorphism. \todo{This definition is incomplete. (I don't see what is missing)} A subset F' $\subseteq F$ is a \emph{retract} of F if there exists a subsitution $\sigma$ such that $\sigma(F) = F'$ and $\sigma_{|F'}=id_{|F'}$ ($\sigma$ is called a \emph{retractation} from F to F'). A factbase is a \emph{core} if all of its strict subsets are not retracts. A \emph{core} of a factbase F is a minimal retract of F that is a core.
\end{definition}

\begin{example}
F = $\{R(a,x)\}$ is the core of F' = $\{R(a,x),R(y,z)\}$ because : 
\begin{itemize}
\item $F \subseteq F'$;
\item $\{x \mapsto x, y \mapsto a, z \mapsto x\}$ is a retractation from F' to F;
\item F is minimal because card(F) = 1 and $\o$ is not a retract of F';
\item F is a core because $\o$ is not a retract of F'.
\end{itemize}
\end{example}

\todo{Explain why; look for a more interesting example.}


\begin{proposal}
A factbase F is a core $\Leftrightarrow$ every homomorphism $\sigma: F \to F$ is a bijection.
\end{proposal}

\begin{proof}
We show it by double-implication. \\
$\boxed{\Leftarrow}$ By contraposition, suppose that the factbase F is not a core : there exists a strict substet F' of F such that F' is a retract of F. There exists a homomorphism $\sigma:F \to F$ such that $\sigma(F) = F'$. As $F' \subsetneq F$, $\sigma$ is not surjective, so it is not a bijection. \\
$\boxed{\Rightarrow}$ Conversely, by contraposition, suppose that there exists an homomorphism $\sigma_1$ not bijective. As F is finite, $\sigma_1$ is not surjective. We pose F' = $\sigma_1(F)\subsetneq F$ and we pose $\sigma_2:F \to F$ such that for $x \in F'$, $\sigma_2(x) = x$ and for $x \notin F'$, $\sigma_2(x) = \sigma_1(x)$. We have ${\sigma_2}_{|F'} = id_{|F'}$ and $\sigma_2(F) = F'$. So $\sigma_2$ is a retractation from F to F' and so F' is a strict retract of F. Consequently F is not a core. It concludes the proof.
\end{proof}


\begin{definition}[existential rule, ontology]
Let $\vec x$, $\vec y$ and $\vec z$ be tuple of variables. An \emph{(existential) rule} R is a first-order formula	 of the form  $\forall \vec x.\forall \vec y. A(\vec x,\vec y) \rightarrow \exists \vec z. B(\vec x,\vec z)$\ where  A and B are conjunctions of atoms. We define \emph{body(R)} = A, \emph{head(R)} = B and the frontier of R \emph{fr(R)} = $\vec x$ (the set of variables shared by the body and the head of R). 

An \emph{ontology} O is a pair (T,F) where T is a set of existential rules and F is a factbase.
\end{definition}

\begin{definition}[trigger]
Let T be a rule set, $\alpha$ be a rule, $\sigma$ be a subsitution and F be a factbase. The tuple $(\alpha,\sigma)$ is a \emph{trigger} for F if : 
\begin{itemize}
\item the domain of $\sigma$ is the set of all variables occurring in Body($\alpha$).
\item $\sigma(Body(\alpha)) \subseteq F$.
\item For all $\hat \sigma$ which extends $\sigma$ and has its domain equals to the set of all variables occurring in Body($\alpha$) and Head($\alpha$) ,$\hat \sigma(Head(\alpha)) \nsubseteq F$
\end{itemize}
\emph{$Tr_{\alpha}(F)$} is the set of all trigers $(\alpha,\sigma)$ for F. \emph{$Tr_{T}(F)$} is the set of all trigers $(\alpha,\sigma)$ for F where $\alpha \in T$ and $\sigma$ is a substitution. 
\end{definition}

\begin{example}If $\alpha = A(x,y) \rightarrow \exists z.B(x,z)$, F = $\{A(b,c)\}$ and $\sigma = \{x \mapsto b, y \mapsto c \}$ then $(\alpha,\sigma)$ is a trigger for F.
\end{example}

\begin{definition}[satisfaction, model, BCQ, universal model]
The rule $\alpha$ is \emph{satisfied} by F if $Tr_{\alpha}(F)$ = $\O$. T is \emph{satisfied} by F if $Tr_{T}(F)$ = $\O$.
a rule set T is \emph{satisfied} by a factbase M if every rule of T is satisfied by M. A factset M is a \emph{model} for an ontology O = (T,F) if $F \subseteq M$ and T is satisfied by M. A \emph{Boolean conjunctive query (BCQ)} is a closed formula of the form \\ $\exists x_1,...,x_n. F(x_1,...,x_n)$ where F is a conjunction of atoms. A fact set F \emph{entails} a BCQ $B = \exists x_1,...,x_n.F(x_1,...,x_n)$ (noted $F \vDash B$) if there exists a substitution $\sigma$ such that $\sigma(B) \subseteq F$. An ontology O \emph{entails} a BCQ $B = \exists x_1,...,x_n.F(x_1,...,x_n)$ (noted $O \vDash B$) if for every model M of O, $M \vDash B$. A model U for an ontology O is \emph{universal} if for
every model M of O, there exists a\todo{Write a instead.}\ homomorphism $h : U \to M$.
\end{definition}

\begin{example} We pose O = $(\{\alpha\},F)$ where $\alpha = A(x,y) \rightarrow \exists z.A(x,z)$ and F = $\{A(b,c)\}$. We pose U = $\{A(b,c)\}\cup \{A(b,x_i)/ i \in \N\}$.
\begin{itemize}
\item $F \subseteq U$
\item $\{\alpha\}$ is satisfied by U
\item let M be a model of O. We construct by induction a sequence$(y_i)_{i \in \N}$ of variables such that $A(b,y_n) \in M$ : 
\begin{itemize}
\item As $F \subseteq M$, $A(b,c) \in M$ and as $\{\alpha\}$ is satisfied by M, there exists a variable $y_0$ such that $A(b,y_0) \in M$
\item if $y_n$ is defined and $A(b,y_n) \in M$, as $\{\alpha\}$ is satisfied by M, there exists a variable $y_{n+1}$ such that $A(b,y_{n+1}) \in M$
\end{itemize}
We then pose 
\begin{align*}
h:U &\to M\\
x_i &\mapsto y_i
\end{align*}
h is a homomorphism from U to M.
\end{itemize}
Consequently, U is a universal model of O.
\end{example}

\begin{definition}[Entailment]
A factbase F \emph{entails} a factbase F' (often noted F $\models F'$) if for all model M de T, M is a model of F'. For example, F = \{P(b,a),Q(x)\} entails F' = \{P(x,a)\} due to the homomorphism $\{x \mapsto b\}$ 
\end{definition}

\begin{proposal}
A factbase F \emph{entails} a factbase F' (often noted F $\models F'$) if and only if there exists a homomorphism from F' to F.
\end{proposal}

\subsection{core chase for finite derivation}

\begin{definition}[application]
Let $\alpha$ be an axiom and F be a factbase. For a trigger $(\alpha,\sigma) \in Tr_\alpha(F)$, the \emph{application} of $(\alpha,\sigma)$ to F is $App_{(\alpha,\sigma)}(F) = F\cup \hat \sigma(Head(\alpha))$ where $\hat \sigma$ extends $\sigma$ and for all $y \notin var(\sigma), \hat \sigma(y) = z_{(\alpha,\sigma)}$ where $z_{(\alpha,\sigma)}$ is a new variable. We pose $App_{\alpha}(F) = \cup_{(\alpha,\sigma) \in Tr_\alpha(F)}App_{(\alpha,\sigma)}(F)$ and $App_{T}(F) = \cup_{\alpha \in T}App_{\alpha}(F)$.

\end{definition} 

\begin{remark} Let $\prec$ be a strict partial order over the set of variables.
\end{remark}

\begin{definition}[prune]


\end{definition} 

\begin{definition}[core chase]
A \emph{core chase sequence} for a rule set T is a sequence $(F_n)_{n \in \N}$ of factbases where : 
\begin{itemize}
\item $F_0 = F$;
\item for all odd numbers $i \in \N^*$, $F_i = App_T(F_{i-1})$.
\item for all even numbers $i \in \N^*$, $F_i = Prune(F_{i-1})$.
\end{itemize}
The core chase \emph{terminates} on T if there exits $i \in \N$ such that $F_{i+1} = F_i$. In this case, we pose \emph{Core(T)} = $F_i$.
\end{definition} 

\section{Horn-ALC}

\end{document}
\begin{comment}A variable x occuring in a factbase F is \emph{superfluous} if there exists two variables y and z such that for all predicate P, ... \\ \\
The \emph{pruning sequence} of a factbase F is the sequence $(F_i)_{i \in \{1,...,n\}}$ of factbases where :
\begin{itemize}
\item $F_1 = F$;
\item for all $i \in \{2,...,n\}$, ...
\item $F_n$ does not contain any superfluous variables.
\end{itemize}
We pose \emph{prune(F)} = $F_n$.
\end{comment}
